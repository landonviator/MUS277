\documentclass{tufte-handout}
\usepackage{amsmath}

\usepackage{graphicx}
\setkeys{Gin}{width=\linewidth,totalheight=\textheight,keepaspectratio}
\graphicspath{{graphics/}}

\title{Project 1: Side-chain Compression}
\author{Dr. Viator}
\date{January 26th, 2026}

% The following package makes prettier tables.  We're all about the bling!
\usepackage{booktabs}

% The units package provides nice, non-stacked fractions and better spacing
% for units.
\usepackage{units}

% The fancyvrb package lets us customize the formatting of verbatim
% environments.  We use a slightly smaller font.
\usepackage{fancyvrb}
\fvset{fontsize=\normalsize}

% Small sections of multiple columns
\usepackage{multicol}

\newcommand{\doccmd}[1]{\texttt{\textbackslash#1}}% command name -- adds backslash automatically
\newcommand{\docopt}[1]{\ensuremath{\langle}\textrm{\textit{#1}}\ensuremath{\rangle}}% optional command argument
\newcommand{\docarg}[1]{\textrm{\textit{#1}}}% (required) command argument
\newenvironment{docspec}{\begin{quote}\noindent}{\end{quote}}% command specification environment
\newcommand{\docenv}[1]{\textsf{#1}}% environment name
\newcommand{\docpkg}[1]{\texttt{#1}}% package name
\newcommand{\doccls}[1]{\texttt{#1}}% document class name
\newcommand{\docclsopt}[1]{\texttt{#1}}% document class option name

\begin{document}
\maketitle

\section{Overview}
In this assignment, you will learn how sidechain compression is used in modern music production to create space and movement in a mix. Using Reason, you will set up a compressor that reacts to an external “trigger” signal—most commonly a kick drum—to automatically reduce the volume of another sound like a bass, pad, or synth.

\section{Steps}
\begin{enumerate}
    \item Pull in an instrument instance of Kong and Monotone into the rack. This should create two new channels in the sequencer.
    \item Pull in an instance of the MClass Compressor under "effects".
    \item Flip to the back of the rack and plug in from the Kong's Aux Send 1 Out left and right channels into the side-chain inputs on the compressor.
    \item Plug in the output of Monotone into the input of the compressor and plugin the output of the compressor into the into of the Monotone's Mix channel.
    \item Flip to the front of the rack and turn up Kong's aux 1 dial to feed signal into the compressor side-chain.
    \item You should now see the meter on the compressor trigger at the same time that Kong's kicks are triggered.
\end{enumerate}

\section{Compression}
A compressor automatically reduces the volume of a signal once it gets too loud, helping control dynamics and make sounds sit better in a mix.

\begin{enumerate}
\item Input: The level going into the compressor. Higher input means the signal hits the compressor harder and triggers more gain reduction.
\item Threshold: The level where compression begins. When the input rises above this point, the compressor starts turning the signal down.
\item Ratio: How strongly the compressor reduces volume after the threshold. For example, a 4:1 ratio means that for every 4 dB above the threshold, only 1 dB comes out.
\item Attack: How quickly the compressor reacts once the signal passes the threshold. Fast attack clamps down immediately; slower attack lets the initial hit through.
\item Release: How quickly the compressor stops compressing after the signal falls back below the threshold. Fast release returns to normal quickly; slow release smooths things out.
\item Output (Makeup Gain): The final volume after compression. Since compression reduces peaks, output gain is used to bring the overall level back up.
\end{enumerate}

\end{document}
